\documentclass[11pt]{article}

    \usepackage{graphicx}
    \usepackage{float}
    \usepackage{caption}
    \usepackage{authblk}
    
    \usepackage[style=authoryear]{biblatex}
    \addbibresource{report.bib}
    \usepackage{hyperref}
    
    \title{RSNA Cervical Spine Fracture Detection}

    \author[1]{Eisha, FNU\thanks{\href{mailto:eisha@uni.minerva.edu}{eisha@uni.minerva.edu}}}
    \author[1]{Hyder, Haitham Alhad\thanks{\href{mailto:haitham.hyder@uni.minerva.edu}{haitham.hyder@uni.minerva.edu} }}
    \author[1]{Wambui, Stevedavies\thanks{\href{mailto:stevedaviesndegwa@uni.minerva.edu}{stevedaviesndegwa@uni.minerva.edu}}}
    \affil[1]{
      \textbf{Haadavand's Minions}  \\
      \texttt{
        \href{mailto:hadavands-minions@uni.minerva.edu}
        {hadavands-minions@uni.minerva.edu}
      }
    }

    \renewcommand\Authands{ and }
    \date{\vspace{-5ex}}


\begin{document}

\maketitle

\section*{Research Question}\label{research-question}

How can we quickly detect and locate vertebral fractures in the 
cervical spine from CT scans to aid in the prevention of 
neurologic deterioration and paralysis after trauma?

\section*{Justification}\label{justification}

In the US alone, there are ~17,730 spinal cord injuries every year \parencite{Morano2021-ak}.
According to the Kaggle competition, most of these spinal cord injuries occur
in the cervical spine (neck) \parencite{kaggle}.
There has also been a rise in the incidences of spinal fractures in the elderly
and this population, fractures can be more difficult to detect on imaging.
Superimposed degenerative disease and osteoporosis is one of the main factors
leading to this difficulty. It is a problem since most of these image
diagnosis are done almost exclusively using computed tomography (CT) \parencite{munera2012}
instead of radiographs (x-rays).

According to the National Spinal Cord Injury Statistical Center
about 39.3\% of these spinal cord injuries are from vehicle crashes.
And another 31.8\% from falls \parencite{NSCISC2019}.
Therefore, quick detection of vertebral fractures and their location is essential
when preventing deterioration and paralysis after trauma. Therefore, through
this project, we will be aiming to first detect vertebral fractures, and second
if possible, the location of cervical spine fractures. The main aim is
to match the radiologists' performance since the dataset we are using was professionally
annotated by Spine radiology specialists from the 
\href{https://www.asnr.org/}{American Society of Neuroradiology (ASNR)}
and the \href{https://www.theassr.org/}{American Society of Spine Radiology (ASSR)}.

\section*{Data Collection Strategies}\label{data-collection-strategies}

For this project, the main source of the dataset is Kaggle where
the image data and labels have already been given. 
This was convenient because we did not have to go through hoops 
of determining whether or not we can scrape data from a given 
website. It also meant that we did not have to spend time figuring 
out what type of data to collect and what is relevant for a given 
scenario. 

However, we did not anticipate the challenge that would come with working 
with over 300 GB of image data with laptops meant for school work. 
This is a challenge that was unexpected and took significantly longer 
o work through than we planned for. Specifically, the breakdown of 
the challenges and how we overcame them is as follows:

\begin{enumerate}
  \item With 300 GB of data, it meant we needed to have at least 
  600 GB of local storage because the data was zipped: the zipped 
  file would be 300 GB and the unzipped file would be 300 GB. 
  Haitham bought an external hard disk with enough storage to store 
  this data and was able to successfully download the data.
  \item The unzipping of the data took a very long time: 
  at least 14 hours. Since this was a long wait time, we also 
  looked into alternatives: we found a jpg dataset with a much
  greater reduction in size (59 GB).
  \item Github has a limit on the size of files you can upload and 
  therefore couldn't upload the data to Github. Instead we uploaded
  the image data to a Google Drive folder (
    \url{https://drive.google.com/drive/u/0/folders/1TbfCraI4R2sVk1jj4_b6NMOFRyPlwAGh}
  ) and we will read the data 
  into a Google Colab file that is in the same directory as the folder 
  of images.
\end{enumerate}

% Add a bibliography block to the postdoc
\printbibliography

\end{document}